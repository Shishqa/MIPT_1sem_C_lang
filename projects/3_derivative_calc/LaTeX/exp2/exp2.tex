\documentclass[a4paper,12pt]{article}
\usepackage{amsmath,amsthm,amssymb}
\usepackage{mathtext}
\usepackage[T1,T2A]{fontenc}
\usepackage[utf8]{inputenc}
\usepackage[english, russian]{babel}
\begin{document}
\section*{Стишок о дифференцировании}\\
Продифференцируем $(2 \cdot x ^{2}  + 3 \cdot x ^{2} )$, ведь мы не деградируем\\
Ну константа - тривиально, и ничуть не криминально\\
\begin{math}
	(2)' = 0
\end{math}\\
Знает рыжая лисица, что у нас тут единица\\
\begin{math}
	(x)' = 1
\end{math}\\
Вниз снеси ты показатель, производной соискатель\\
\begin{math}
	(x ^{2} )' = 2 \cdot x ^{1}  \cdot 1
\end{math}\\
Скобок мельтешение - раскрыли умножение\\
\begin{math}
	(2 \cdot x ^{2} )' = 0 \cdot x ^{2}  + 2 \cdot 2 \cdot x ^{1}  \cdot 1
\end{math}\\
Ну константа - тривиально, и ничуть не криминально\\
\begin{math}
	(3)' = 0
\end{math}\\
Знает рыжая лисица, что у нас тут единица\\
\begin{math}
	(x)' = 1
\end{math}\\
Вниз снеси ты показатель, производной соискатель\\
\begin{math}
	(x ^{2} )' = 2 \cdot x ^{1}  \cdot 1
\end{math}\\
Скобок мельтешение - раскрыли умножение\\
\begin{math}
	(3 \cdot x ^{2} )' = 0 \cdot x ^{2}  + 3 \cdot 2 \cdot x ^{1}  \cdot 1
\end{math}\\
Производная суммы, тут ничего не рифмуется\\
\begin{math}
	(2 \cdot x ^{2}  + 3 \cdot x ^{2} )' = 0 \cdot x ^{2}  + 2 \cdot 2 \cdot x ^{1}  \cdot 1 + 0 \cdot x ^{2}  + 3 \cdot 2 \cdot x ^{1}  \cdot 1
\end{math}\\
\\Поумерь, дружочек, злобу. Получили зелибобу\\ $(2 \cdot x ^{2}  + 3 \cdot x ^{2} )' = 0 \cdot x ^{2}  + 2 \cdot 2 \cdot x ^{1}  \cdot 1 + 0 \cdot x ^{2}  + 3 \cdot 2 \cdot x ^{1}  \cdot 1$\\
Чтобы похвастаться тёще, получим штуку попроще\\
$(2 \cdot x ^{2}  + 3 \cdot x ^{2} )' = 2 \cdot 2 \cdot x + 3 \cdot 2 \cdot x$\\
\end{document}

