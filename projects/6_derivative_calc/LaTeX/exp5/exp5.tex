\documentclass[a4paper,12pt]{article}
\usepackage{amsmath,amsthm,amssymb}
\usepackage{mathtext}
\usepackage[T1,T2A]{fontenc}
\usepackage[utf8]{inputenc}
\usepackage[english, russian]{babel}
\begin{document}
\section*{Стишок о дифференцировании}\\
Продифференцируем $( sh (2 \cdot x + 3)  ^{2} )$, ведь мы не деградируем\\
Ну константа - тривиально, и ничуть не криминально\\
\begin{math}
	(2)' = 0
\end{math}\\
Знает рыжая лисица, что у нас тут единица\\
\begin{math}
	(x)' = 1
\end{math}\\
Скобок мельтешение - раскрыли умножение\\
\begin{math}
	(2 \cdot x)' = 0 \cdot x + 2 \cdot 1
\end{math}\\
Ну константа - тривиально, и ничуть не криминально\\
\begin{math}
	(3)' = 0
\end{math}\\
Производная суммы, тут ничего не рифмуется\\
\begin{math}
	((2 \cdot x + 3))' = 0 \cdot x + 2 \cdot 1 + 0
\end{math}\\
\begin{math}
	( sh (2 \cdot x + 3) )' =  ch (2 \cdot x + 3)  \cdot (0 \cdot x + 2 \cdot 1 + 0)
\end{math}\\
Вниз снеси ты показатель, производной соискатель\\
\begin{math}
	( sh (2 \cdot x + 3)  ^{2} )' = 2 \cdot  sh (2 \cdot x + 3)  ^{1}  \cdot  ch (2 \cdot x + 3)  \cdot (0 \cdot x + 2 \cdot 1 + 0)
\end{math}\\
\\Поумерь, дружочек, злобу. Получили зелибобу\\ $( sh (2 \cdot x + 3)  ^{2} )' = 2 \cdot  sh (2 \cdot x + 3)  ^{1}  \cdot  ch (2 \cdot x + 3)  \cdot (0 \cdot x + 2 \cdot 1 + 0)$\\
Чтобы похвастаться тёще, получим штуку попроще\\
$( sh (2 \cdot x + 3)  ^{2} )' = 2 \cdot  sh (2 \cdot x + 3)  \cdot 2 \cdot  ch (2 \cdot x + 3) $\\
\end{document}

