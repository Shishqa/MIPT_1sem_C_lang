\documentclass[a4paper,12pt]{article}
\usepackage{amsmath,amsthm,amssymb}
\usepackage{mathtext}
\usepackage[T1,T2A]{fontenc}
\usepackage[utf8]{inputenc}
\usepackage[english, russian]{babel}
\begin{document}
\section*{Стишок о дифференцировании}\\
Продифференцируем $(( \frac{ sin x }{x}  + x ^{10} ) \cdot (x + 2 \cdot x))$, ведь мы не деградируем\\
Знает рыжая лисица, что у нас тут единица\\
\begin{math}
	(x)' = 1
\end{math}\\
Синус быстренько раскроем, а потом полы помоем\\
\begin{math}
	( sin x )' =  cos x  \cdot 1
\end{math}\\
Знает рыжая лисица, что у нас тут единица\\
\begin{math}
	(x)' = 1
\end{math}\\
Производная частного для тебя несчастного\\
\begin{math}
	( \frac{ sin x }{x} )' =  \frac{( cos x  \cdot 1 \cdot x -  sin x  \cdot 1)}{x ^{2} } 
\end{math}\\
Знает рыжая лисица, что у нас тут единица\\
\begin{math}
	(x)' = 1
\end{math}\\
Вниз снеси ты показатель, производной соискатель\\
\begin{math}
	(x ^{10} )' = 10 \cdot x ^{9}  \cdot 1
\end{math}\\
Производная суммы, тут ничего не рифмуется\\
\begin{math}
	(( \frac{ sin x }{x}  + x ^{10} ))' =  \frac{( cos x  \cdot 1 \cdot x -  sin x  \cdot 1)}{x ^{2} }  + 10 \cdot x ^{9}  \cdot 1
\end{math}\\
Знает рыжая лисица, что у нас тут единица\\
\begin{math}
	(x)' = 1
\end{math}\\
Ну константа - тривиально, и ничуть не криминально\\
\begin{math}
	(2)' = 0
\end{math}\\
Знает рыжая лисица, что у нас тут единица\\
\begin{math}
	(x)' = 1
\end{math}\\
Скобок мельтешение - раскрыли умножение\\
\begin{math}
	(2 \cdot x)' = 0 \cdot x + 2 \cdot 1
\end{math}\\
Производная суммы, тут ничего не рифмуется\\
\begin{math}
	((x + 2 \cdot x))' = 1 + 0 \cdot x + 2 \cdot 1
\end{math}\\
Скобок мельтешение - раскрыли умножение\\
\begin{math}
	(( \frac{ sin x }{x}  + x ^{10} ) \cdot (x + 2 \cdot x))' = ( \frac{( cos x  \cdot 1 \cdot x -  sin x  \cdot 1)}{x ^{2} }  + 10 \cdot x ^{9}  \cdot 1) \cdot (x + 2 \cdot x) + ( \frac{ sin x }{x}  + x ^{10} ) \cdot (1 + 0 \cdot x + 2 \cdot 1)
\end{math}\\
\\Поумерь, дружочек, злобу. Получили зелибобу\\ $(( \frac{ sin x }{x}  + x ^{10} ) \cdot (x + 2 \cdot x))' = ( \frac{( cos x  \cdot 1 \cdot x -  sin x  \cdot 1)}{x ^{2} }  + 10 \cdot x ^{9}  \cdot 1) \cdot (x + 2 \cdot x) + ( \frac{ sin x }{x}  + x ^{10} ) \cdot (1 + 0 \cdot x + 2 \cdot 1)$\\
Чтобы похвастаться тёще, получим штуку попроще\\
$(( \frac{ sin x }{x}  + x ^{10} ) \cdot (x + 2 \cdot x))' = ( \frac{( cos x  \cdot x -  sin x )}{x ^{2} }  + 10 \cdot x ^{9} ) \cdot (x + 2 \cdot x) + 3 \cdot (x ^{10}  +  \frac{ sin x }{x} )$\\
\end{document}

