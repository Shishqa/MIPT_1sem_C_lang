\documentclass[a4paper,12pt]{article}
\usepackage{amsmath,amsthm,amssymb}
\usepackage{mathtext}
\usepackage[T1,T2A]{fontenc}
\usepackage[utf8]{inputenc}
\usepackage[english, russian]{babel}
\begin{document}
\section*{Стишок о дифференцировании}\\
Продифференцируем $(3 \cdot x + 10 \cdot x + x ^{x} )$, ведь мы не деградируем\\
Ну константа - тривиально, и ничуть не криминально\\
\begin{math}
	(3)' = 0
\end{math}\\
Знает рыжая лисица, что у нас тут единица\\
\begin{math}
	(x)' = 1
\end{math}\\
Скобок мельтешение - раскрыли умножение\\
\begin{math}
	(3 \cdot x)' = 0 \cdot x + 3 \cdot 1
\end{math}\\
Ну константа - тривиально, и ничуть не криминально\\
\begin{math}
	(10)' = 0
\end{math}\\
Знает рыжая лисица, что у нас тут единица\\
\begin{math}
	(x)' = 1
\end{math}\\
Скобок мельтешение - раскрыли умножение\\
\begin{math}
	(10 \cdot x)' = 0 \cdot x + 10 \cdot 1
\end{math}\\
Производная суммы, тут ничего не рифмуется\\
\begin{math}
	(3 \cdot x + 10 \cdot x)' = 0 \cdot x + 3 \cdot 1 + 0 \cdot x + 10 \cdot 1
\end{math}\\
Знает рыжая лисица, что у нас тут единица\\
\begin{math}
	(x)' = 1
\end{math}\\
Знает рыжая лисица, что у нас тут единица\\
\begin{math}
	(x)' = 1
\end{math}\\
Мама Люба раму мыла, щас получим крокодила\\
\begin{math}
	(x ^{x} )' = x ^{x}  \cdot (1 \cdot  ln x  +  \frac{1}{x}  \cdot x)
\end{math}\\
Производная суммы, тут ничего не рифмуется\\
\begin{math}
	(3 \cdot x + 10 \cdot x + x ^{x} )' = 0 \cdot x + 3 \cdot 1 + 0 \cdot x + 10 \cdot 1 + x ^{x}  \cdot (1 \cdot  ln x  +  \frac{1}{x}  \cdot x)
\end{math}\\
\\Поумерь, дружочек, злобу. Получили зелибобу\\ $(3 \cdot x + 10 \cdot x + x ^{x} )' = 0 \cdot x + 3 \cdot 1 + 0 \cdot x + 10 \cdot 1 + x ^{x}  \cdot (1 \cdot  ln x  +  \frac{1}{x}  \cdot x)$\\
Чтобы похвастаться тёще, получим штуку попроще\\
$(3 \cdot x + 10 \cdot x + x ^{x} )' = 13 + x ^{x}  \cdot ( ln x  +  \frac{1}{x}  \cdot x)$\\
\end{document}

