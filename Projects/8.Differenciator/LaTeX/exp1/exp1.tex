\documentclass[a4paper,12pt]{article}
\usepackage{amsmath,amsthm,amssymb}
\usepackage{mathtext}
\usepackage[T1,T2A]{fontenc}
\usepackage[utf8]{inputenc}
\usepackage[english, russian]{babel}
\begin{document}
\section*{Стишок о дифференцировании}\\
Продифференцируем $( sin 1.50  \cdot x)$, ведь мы не деградируем\\
Ну константа - тривиально, и ничуть не криминально\\
\begin{math}
	(1.50)' = 0
\end{math}\\
Синус быстренько раскроем, а потом полы помоем\\
\begin{math}
	( sin 1.50 )' =  cos 1.50  \cdot 0
\end{math}\\
Знает рыжая лисица, что у нас тут единица\\
\begin{math}
	(x)' = 1
\end{math}\\
Скобок мельтешение - раскрыли умножение\\
\begin{math}
	( sin 1.50  \cdot x)' =  cos 1.50  \cdot 0 \cdot x +  sin 1.50  \cdot 1
\end{math}\\
\\Поумерь, дружочек, злобу. Получили зелибобу\\ $( sin 1.50  \cdot x)' =  cos 1.50  \cdot 0 \cdot x +  sin 1.50  \cdot 1$\\
Чтобы похвастаться тёще, получим штуку попроще\\
$( sin 1.50  \cdot x)' = 1$\\
\end{document}

